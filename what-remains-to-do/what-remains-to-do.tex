\documentclass[a4paper,twoside,12pt]{article}
%\documentclass[a4paper,twoside,draft,12pt]{article}

\def\AUTHOR{Luca Saiu\xspace}
\def\TITLE{Multi-runtime OCaml\xspace}
\def\SUBTITLE{what remains to do\xspace}
\def\SHORTTITLE{\TITLE -- \SUBTITLE}

\include{format-and-defs}

\renewcommand{\SECTION}[1]{§\ref{#1}}
\newcommand{\MULTIRUNTIME}[0]{\textit{multi-runtime}\xspace}
\newcommand{\EMAIL}[1]{\href{mailto:{#1}}{<\texttt{#1}>}}
%\newcommand{\EMAIL}[1]{\url{ageinghacker.net} \href{http://www.gnu.org}{<{#1}>}}
\newcommand{\BLAHS}[0]{Blah blah blah blah blah blah blah blah. }
\newcommand{\TEXT}[0]{\BLAHS \BLAHS \BLAHS \BLAHS \BLAHS \BLAHS \BLAHS \BLAHS \BLAHS \BLAHS \BLAHS \BLAHS \BLAHS}

\author{\AUTHOR \EMAIL{positron@gnu.org}}
\title{\TITLE\ ---\ \SUBTITLE}
\date{2013-09-27}

\begin{document}
\maketitle

% ===================================================
The core ideas should be quite easy to understand from my presentation
slides I used in June 2013 when speaking to the BWare group:
\\
{\small
  \url{http://ageinghacker.net/talks/ocaml-multiruntime-presentation.pdf}}
\\
The user-visible implementation state as described there is
up-to-date, with the following exceptions:
\begin{itemize}
\item
  the \MULTIRUNTIME system now works with \textit{systhreads};
\item
  the \MULTIRUNTIME system can be disabled at configuration time;
\item
  we also support (without \MULTIRUNTIME) PowerPC and i386 GNU/Linux;
  non-\MULTIRUNTIME support for the other architectures is trivial to add (see \SECTION{other-architectures})
\item
  there's one new crucial optimization for non-\MULTIRUNTIME
  configurations, allowing to avoid an context indirection when
  referring a contextual runtime variable, thanks to the conditional
  definition of \CODE{ctx}.
\end{itemize}


% ===================================================
\section{A few debugging tips}
By far the hardest problems to debug are relating to forgetting to
notify the collector about Caml GC roots in C code, and having one
context objects reachable from objects belonging to a different
context; the result of these mistakes is some apparent memory
corruption, happening far from its cause both in space and time.

The good way of debugging this is via \CODE{gdb}, but you have to
reason \textit{forward} (using breakpoints) rather than
\textit{backward}; when a failure manifests itself, it's usually far
too late to discover its cause back in time.

A crude debugging technique which works is to tentatively comment-out
parts of the C code, until the failure stops manifesting itself.  Of
course we can use prints from the code; I defined a relatively complex
printing facility for debugging in \CODE{runtime/context.h} (look for
\CODE{DUMP}), using variadic macros.
\\
\\
Many failures are non-deterministic, so it's useful to have some
automatic facility to run a test many times, and see if some run fails
instead of either exiting with success, or being still alive after a
given timeout.

Here's my disgusting \CODE{bash} function doing this (interrupting the function
leaves dangling processes):
\begin{Verbatim}
# Really, really horrible.  FIXME: remove $directory if interrupted.  Omit
# job status output
function crash_after_times
{
    noncrash_timeout="$1"; shift
    times="$1"; shift
    command="$@"; shift
    echo "command is \"$command\""
    for (( i=1; $i <= $times; i++ )); do
        echo "Iteration $i of $times:"
        directory="$(mktemp -d)"
        successfile="$directory/success"
        sleep_pid=$!
        (bash -c "$command" &> /dev/null && touch "$successfile") &
        pid=$!
        sleep "$noncrash_timeout"
        if kill -0 "$pid" &> /dev/null; then
            kill -KILL "$pid" &> /dev/null || true
            echo "* iteration $i: still alive after $noncrash_timeout seconds."
            rm -rf "$directory"
        elif [ -f "$successfile" ]; then
            echo "* iteration $i: succeeded before $noncrash_timeout seconds."
            rm -rf "$directory"
        else
            echo "* iteration $i: failed before $noncrash_timeout seconds."
            echo "FAILURE (after $i attempts)"
            rmdir "$directory"
            return 255
        fi
        # Kill processes named after the first part of the command
        # line, up to the first space.  Yes, I know, this sucks.
        killall $(echo "$command" | awk -F ' ' '{print $1}') &> /dev/null || true

        # kill -KILL "$pid" &> /dev/null || true
    done
    echo "SUCCESS: no crashes before $noncrash_timeout seconds after $times attempts."
    return 0
}
\end{Verbatim}

\CODE{valgrind} is your friend;  \CODE{helgrind}, not nearly as much: too many false positives.
\\
\\
\textit{vmthreads} use \CODE{SIGALRM} for preemption at time thread
switching time.  Beware of primitives interrupted by signals.  A
\CODE{sem\_wait} operation unexpectedly interrupted by a signal is a
funny thing to debug, so to speak.

% ===================================================
\section{Grunt work, mostly cosmetic}
Some trivial tasks remain to do, which I didn't finish only for lack of time:
\begin{itemize}
\item free the context structure field at context destruction time;
\item make mailboxes a \textit{custom} type; both mailboxes and
  contexts are currently pointers to \CODE{malloc}'ed objects tagged
  as unboxed caml objects --- the current solution is fine for
  contexts because they are finalized in a special way, but mailboxes
  need real finalization at GC time;
\item remove my horrible debug prints, commented-out alternate code
  and tests, and forced \CODE{caml\_gc\_compaction\_r} code
  which pollute C code everywhere;
\item possibly rename the Caml module \CODE{Context}.  Fabrice
  suggested \CODE{Runtime}, but I'd propose the more explicit
  \CODE{MultiContext}, \CODE{MultiRuntime}, or possibly just \CODE{Multi};
\item support \CODE{otherlibs/graphics};
\item support \textit{Dynlink} (more interesting).
\end{itemize}

% ===================================================
\section{Nitpicking}
When explaining the source code to Xavier and Damien, I told them
about a potential problem: in the current implementation if a split
operation (say, a \CODE{pthread\_create} call) fails because of
unavailable resources \textit{after} the blob has been allocated, it's
not obvious how to have the Caml \CODE{Context.split} operation fail cleanly.
In that case at the very least \textit{we leak the blob}, but effects
are potentially more dangerous.  In order to provide a clean and safe
interface, we could add a further synchronization phase in which the
parent context waits for all children to have split with success,
before returning to the caller; if any of them failed, the parent has
to ask all non-failing children to rollback and die, so as to free all
resources before ever starting to execute Caml code.  Only if all
children performed their part of the split operation with success, the
parent can authorize them to go on and execute the code.
\\
\\
Xavier and Damien seemed to take this issue seriously and I understand
their stance from a philosophical point of view, but I'm not sure if
this complexity will be worth the trouble in practice: in the use
cases we care about splitting will happen early in the program
execution, and the number of contexts will never be much larger than
the number of CPU cores.

I think that the extra-safe version can be implemented later, and the
current splitting code will not fail for lack of resources in any
reasonable scenario.
By the way, even if some commented-out code remains from previous
tests, the splitting logic is actually pretty simple.  We can add a
new synchronization phase.  It won't be critical for performance, in
the common use case.

% ===================================================
\section{OCaml thread cleanup}
Xavier was considering the idea of removing vmthreads altogether, and
requiring some form of threads in the main OCaml runtime, without
external libraries.  I'd strongly support that.

Vmthreads are currently incompatible with multicontext (I started to
add vmthreads support in scheduler.c, but some bug remains) and
considerably complex because of blocking behaviour.  Even more
importantly, the interaction of threads with the rest of the runtime
is complex, for historical reasons.  The code in \FILE{byterun/} and
\FILE{asmrun/} is supposed to work with no threads, systhreads or
vmthreads: the initialization function for a threading library is a
global operation installing some hooks (for GC or locking), which
become valid from that point on.  A good example of the complexity
introduced by this mechanism is the presence of mutexes in channel
structures (\FILE{byterun/io.c}): a per-channel mutex is created
lazily when attempting a lock operations, so we cannot suppose its
presence or absence in general.

If this part of the code is streamlined (and it can be: we can make
thread creation always fail on some platforms) I suggest to proceed
the following way:

\begin{itemize}
\item Implement a thin abstraction layer over system threads: thread
  creation, and operations over synchronization data structures;
\item update my context implementation to use the new layer instead of
  POSIX threads;
\item change systhreads to use the abstraction layer (this part will
  be the most difficult to debug, from my past experience).
\end{itemize}

At that point the existing runtime code can be simplified, and the
multi-context system gains portability.

% ===================================================
\section{Porting to other architectures}
\label{other-architectures}
I had to do some very small changes in \textit{ocamlopt} (essentially
to propagate information down to intermediate representations), which
require a very minor fix in architecture-dependent files.  This is
actually very easy and can be guided by compile-time errors; if you
want an example, look at what I did for the PowerPC and i386 runtime
(the Caml part --- the assembly part is completely untouched).

The important file should be the architecture-dependent \FILE{cmmgen.ml}
\\
\\
The architectures requiring fixes are Arm and Sparc (I already fixed
i386 and PowerPC; amd64 supports \MULTIRUNTIME on GNU/Linux); the
systems are windows and mac os.

% ===================================================
\section{Porting the patch to the OCaml mainline}
My work is based on a snapshot from October 2012.  It has to be ported
to the new mainline, but I'm sure it won't be difficult this time ---
there was an ABI change in the time between the original work by
Fabrice and the beginning of my time at Inria --- see the BWare
slides.  Since there were no such change in recent times, I don't
anticipate any problems.  Most of my work is on the C part, with
smaller changes in the assembly runtime; the OCaml part, which I guess
is the most rapidly-changing one nowadays, is nearly untouched.

\label{on-ageinghacker-net}
I've uploaded the unpatched snapshot I started from on my personal
server, at \url{http://ageinghacker.net/inria-temporary/}.  You'll
also find there my test programs, most of which are ugly and not
really fitting in the repository, and a snapshot of my current git
repo at \url{https://github.com/lucasaiu/ocaml}.

% ===================================================
\section{API compatibility}
We discussed API compatibility with Xavier and Damien.  In order to
provide the old interface with no ``\CODE{\_r}'' suffixes and no
context-pointer parameters, one solution is changing many function
prototypes to use a macro, as exemplified here:
\begin{Verbatim}
#define CAMLFUNCTION1(TYPE, NAME, ARG1TYPE, ARG1NAME) \
  TYPE NAME(ARG1TYPE ARG1NAME){ return NAME##_r(caml_get_thread_local_context(), \
                                                ARG1NAME); } \
  TYPE NAME##_r(CAML_R, ARG1TYPE, ARG1NAME)
\end{Verbatim}
It's ugly, and we also need a special case for the type \CODE{void}
(in C it's invalid to \CODE{return} a \CODE{void}-typed expression,
even from a \CODE{void}-returning procedure).  Xavier and Damien
seemed to find the alternative of explicitly duplicating prototypes
preferable to preprocessor-based solutions; personally I'd always
choose syntactic abstraction over code duplication, but as a Lisper I
understand that we have different cultural biases.

Whatever the choice, a form of backward compatibility will have to
be implemented.

By contrast the ABI will break, and there's nothing to be done about
it.  I don't think it's worth even considering the issue.
\\
\\
This is easy grunt work, but it still has to be done very carefully:
bugs related to parameter names, for example, will be difficult to find.


% ===================================================
\section{Behaviors to be clearly defined}
We don't really know what the good semantics would be for a few
features, for example context finalization in a \textit{multi-thread}
setting..  The whole idea of being able to link several C libraries
using OCaml internally together in the same executable is also still a
little nebulous in its details.

Related to C libraries, there is some support for C
\textit{contextual} variables: the code is there (shared with
contextual Caml variables for native code), but we have to check that
this is the semantics we want.

% ===================================================
\section{Long-term developments}
Longer-term developments include adding a third \textit{ancient}
generation, shared among contexts.  Details are still vague.

\end{document}
